\documentclass[a4paper,12pt]{article}
\usepackage{amssymb} 
\usepackage[utf8]{inputenc}
\usepackage[spanish]{babel}
\usepackage{graphicx}
\usepackage{multicol}
\usepackage{listings}
\usepackage{cprotect}


\makeindex
\begin{document}


\vspace{2cm}
\includegraphics[scale=0.75]{logo_fic_185.png}\hspace{5cm} 
\includegraphics[scale=0.35]{logo-udc.jpg}\\
[2cm]
\begin{center}
  \Large \textbf{Producers and consumers}\\ VV: Exercise 2, model checking
\end{center}


\vspace{10cm}
\begin{flushright}
Denís Graña Fernández \\
José Antonio López Sebio \\
[2cm]
\end{flushright}

\newpage
\tableofcontents
\newpage

\section{Introducción}


\section{Algoritmo naive}

\lstset{
	language=C,
	tabsize=2,  
	showspaces=false,     
	breakatwhitespace=false,
	showstringspaces=false
	}          % Set your language (you can change the language for each code-block optionally)

\begin{lstlisting}[frame=single]  % Start your code-block

#define N 255
byte items = 0;

active [2] proctype Producer(){
	do
	::	items < N;
		printf("Produce nuevo elemento\n");
		cs:printf("%d Elementos en buffer\n\n",items);
		items++;
		
	od;
}

active [1] proctype Consumer(){
	do
	::	 
		items > 0;
		printf("Extraido elemento del buffer\n");
		cs:printf("%d Elementos en buffer\n\n",items);
		assert(items>0);
		items--;
	od;
}
\end{lstlisting}

\subsection{Mutual exclusion}
Para probar la existencia de \textit{mutual exclusion} en el agoritmo \textit{naive}, utilizamos la siguiente fórmula LTL

\begin{lstlisting}
spin -a -f '<>(Consumer[0]@cs && Consumer[1]@cs)' 
producers_consumers.pml
\end{lstlisting}

En caso de realizar la misma prueba con 1 \textit{consumer} y 0 \textit{producer}, o viceversa, no existe \textit{mutual exclusion} ya que sólo hay un proceso corriendo al mismo tiempo.

Para tamaño de buffer=0 queda demostrado que no existe \textit{mutual exclusion} puesto que ningún proceso produce, ni consume nada, ya que no entran en zona crítica.

\subsection{Deadlock}

\begin{lstlisting}
spin -a producers_consumers.pml
gcc -o pan pan.c 
./pan
\end{lstlisting}

\subsection{Starvation}

Para la detección de \textit{starvation} fue usada la siguiente fórmula LTE: 

\begin{lstlisting}
spin -a -f '![]<>(Producer@cs
)' producers_consumers.pml
gcc -o pan pan.c
./pan -a
\end{lstlisting}

Dónde comprobamos que:

\begin{itemize}
\item Con N=0, sí encuentra \textit{starvation}.
\item \textit{producer}=\textit{consumer}=1 no encuentra, puesto que se turnan para producir/consumir.
\item \textit{producer}\textgreater 1 o \textit{consumer}\textgreater 1 sí hay \textit{starvation}.
\end{itemize}

\end{document}          
