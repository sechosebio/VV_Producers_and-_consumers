\documentclass[a4paper,12pt]{article}
\usepackage{amssymb} 
\usepackage[utf8]{inputenc}
\usepackage[spanish]{babel}
\usepackage{graphicx}
\usepackage{multicol}
\usepackage{listings}
\usepackage{cprotect}


\makeindex
\begin{document}


\vspace{2cm}
\includegraphics[scale=0.75]{logo_fic_185.png}\hspace{5cm} 
\includegraphics[scale=0.35]{logo-udc.jpg}\\
[2cm]
\begin{center}
  \Large \textbf{Producers and consumers}\\ VV: Exercise 2, model checking
\end{center}


\vspace{10cm}
\begin{flushright}
Denís Graña Fernández \\
José Antonio López Sebio \\
[2cm]
\end{flushright}

\newpage
\tableofcontents
\newpage

\section{Introducción}


\section{Algoritmo naive}

\lstset{
	language=C,
	tabsize=2,  
	showspaces=false,     
	breakatwhitespace=false,
	showstringspaces=false
	}          % Set your language (you can change the language for each code-block optionally)

\begin{lstlisting}[frame=single]  % Start your code-block

#define N 255
byte items = 0;

active [2] proctype Producer(){
	do
	::	items < N;
		printf("Produce nuevo elemento\n");
		cs:printf("%d Elementos en buffer\n\n",items);
		items++;
		
	od;
}

active [1] proctype Consumer(){
	do
	::	 
		items > 0;
		printf("Extraido elemento del buffer\n");
		cs:printf("%d Elementos en buffer\n\n",items);
		assert(items>0);
		items--;
	od;
}
\end{lstlisting}

\subsection{Mutual exclusion}
Para probar la existencia de \textit{mutual exclusion} en el agoritmo \textit{naive}, utilizamos la siguiente fórmula LTL

\begin{lstlisting}
spin -a -f '<>(Consumer[0]@cs && Consumer[1]@cs)' 
producers_consumers.pml
\end{lstlisting}

En caso de realizar la misma prueba con 1 \textit{consumer} y 0 \textit{producer}, o viceversa, no existe \textit{mutual exclusion} ya que sólo hay un proceso corriendo al mismo tiempo.

Para tamaño de buffer=0 queda demostrado que no existe \textit{mutual exclusion} puesto que ningún proceso produce, ni consume nada, ya que no entran en zona crítica.

\subsection{Deadlock}

\begin{lstlisting}
spin -a producers_consumers.pml
gcc -o pan pan.c 
./pan
\end{lstlisting}

\subsection{Starvation}

Para la detección de \textit{starvation} fue usada la siguiente fórmula LTE: 

\begin{lstlisting}
spin -a -f '![]<>(Producer@cs
)' producers_consumers.pml
gcc -o pan pan.c
./pan -a
\end{lstlisting}

Dónde comprobamos que:

\begin{itemize}
\item Con N=0, sí encuentra \textit{starvation}.
\item \textit{producer}=\textit{consumer}=1 no encuentra, puesto que se turnan para producir/consumir.
\item \textit{producer}\textgreater 1 o \textit{consumer}\textgreater 1 sí hay \textit{starvation}.
\end{itemize}

\newpage
\section{Algoritmo con Semáforos}

\begin{lstlisting}[frame=single]  % Start your code-block

#define N 255
byte holes = N;
byte items = 0;
bool mutex = false;

active [2] proctype Producer(){
        do
        ::
                printf("Produce nuevo item\n");
                /* wait holes */
                atomic{
                        holes;
                        items<N;
                        holes--;
                };
                /* wait mutex */
                atomic{
                        !mutex;
                        mutex = true;
                };
                cs:printf("Escribe en el buffer\n");
                /* signal mutex */
                mutex = false;
                /* signal items */
                items++;

        od;
}

active [2] proctype Consumer(){
        do
        ::
                /* lock items */
                atomic{
                        items;
                        holes<N;
                        items--;
                };
                /* lock mutex */
                atomic{
                        !mutex;
                        mutex = true;
                };
                cs:printf("Extrae elemento del buffer\n");
                /* signal mutex */
                mutex = false;
                /*signal holes */
                holes++;
                printf("Consume elemento\n");
        od;
}
\end{lstlisting}

\subsection{Mutual exclusion}

\subsubsection{Producer, Consumer $>=$ 1}
Para probar la existencia de \textit{mutual exclusion} en el agoritmo \textit{con semáforos}, utilizamos la siguiente fórmula LTL en el caso general, con varios procesos

\begin{lstlisting}
spin -a -f '<>(Producer[0]@cs && Consumer[2]@cs)' producers_consumers.pml
gcc -o pan pan.c -DSAFETY
./pan
\end{lstlisting}
con la siguiente salida:
\begin{lstlisting}[frame=single]  % Start your code-block

...
State-vector 52 byte, depth reached 9999, errors: 0
    27724 states, stored
    43247 states, matched
    70971 transitions (= stored+matched)
    16566 atomic steps
hash conflicts:        11 (resolved)
...
\end{lstlisting}
\textbf{No se encuentran errores, podemos asegurar que no hay mutual exclusion}
Nota: Cuando se hacia p(mutex) sin ser de forma atómica, si que había exclusión mútua

\subsubsection{Un solo proceso}
Cambiamos el código a active [0] proctype y active [1] proctype respectivamente
\begin{lstlisting}
spin -a -f '<>(Producer@cs && Consumer@cs)' producers_consumers.pml
gcc -o pan pan.c -DSAFETY
./pan
\end{lstlisting}

\begin{lstlisting}[frame=single]  % Start your code-block

...
State-vector 28 byte, depth reached 3318, errors: 0
     1532 states, stored
        1 states, matched
     1533 transitions (= stored+matched)
      255 atomic steps
hash conflicts:         0 (resolved)
...
\end{lstlisting}
\textbf{Tampoco hay exclusión mútua}

\subsubsection{Tamaño de buffer 0}
\begin{lstlisting}[frame=single]  % Start your code-block

...
State-vector 52 byte, depth reached 5, errors: 0
        3 states, stored
        1 states, matched
        4 transitions (= stored+matched)
        0 atomic steps
hash conflicts:         0 (resolved)
...
\end{lstlisting}
\textbf{No hay ningún proceso que pase por sección crítica nunca, no hay exclusión mútua}

\subsection{Deadlock}
\subsubsection{Producer, Consumer $>=$ 1}

\begin{lstlisting}
spin -a producers_consumers.pml
gcc -o pan pan.c 
./pan
\end{lstlisting}

\begin{lstlisting}[frame=single]  % Start your code-block

...
State-vector 44 byte, depth reached 9999, errors: 0
    19546 states, stored
    30293 states, matched
    49839 transitions (= stored+matched)
     9917 atomic steps
hash conflicts:       192 (resolved)
...
\end{lstlisting}
\textbf{No hay deadlock}

\subsubsection{Buffer N de tamaño 0}

\begin{lstlisting}
spin -a producers_consumers.pml
gcc -o pan pan.c 
./pan
\end{lstlisting}

\begin{lstlisting}[frame=single]  % Start your code-block

...
State-vector 44 byte, depth reached 2, errors: 1
        3 states, stored
        0 states, matched
        3 transitions (= stored+matched)
        0 atomic steps
hash conflicts:         0 (resolved)
\end{lstlisting}
\textbf{Como el tamaño del buffer es 0, ambos procesos se quedan esperando indefinidamente}
\subsection{Starvation}

Para la detección de \textit{starvation} fue usada la siguiente fórmula LTL: 

\begin{lstlisting}
spin -a -f '!<>(Producer@cs)' producers_consumers.pml
gcc -o pan pan.c
./pan -a
\end{lstlisting}

Dónde comprobamos que:

\begin{lstlisting}[frame=single]  % Start your code-block

...
State-vector 60 byte, depth reached 31, errors: 1
       15 states, stored
        0 states, matched
       15 transitions (= stored+matched)
        2 atomic steps
hash conflicts:         0 (resolved)
...
\end{lstlisting}
\textbf{Se incurre en Starvation, como podemos comprobar en la siguiente traza}

\begin{lstlisting}[frame=single]  % Start your code-block

spin -t -p producers_consumers.pml

starting claim 2
spin: couldn't find claim 2 (ignored)
using statement merging
              Produce nuevo item
  2:	proc  1 (Producer:1) producers_consumers.pml:12 (state 1)	[printf('Produce nuevo item\\n')]
          Produce nuevo item
  4:	proc  0 (Producer:1) producers_consumers.pml:12 (state 1)	[printf('Produce nuevo item\\n')]
  <<<<<START OF CYCLE>>>>>
  6:	proc  1 (Producer:1) producers_consumers.pml:15 (state 2)	[(holes)]
  7:	proc  1 (Producer:1) producers_consumers.pml:16 (state 3)	[((items<255))]
  7:	proc  1 (Producer:1) producers_consumers.pml:17 (state 4)	[holes = (holes-1)]
  9:	proc  1 (Producer:1) producers_consumers.pml:21 (state 6)	[(!(mutex))]
  9:	proc  1 (Producer:1) producers_consumers.pml:22 (state 7)	[mutex = 1]
              Escribe en el buffer
 11:	proc  1 (Producer:1) producers_consumers.pml:24 (state 9)	[printf('Escribe en el buffer\\n')]
 13:	proc  1 (Producer:1) producers_consumers.pml:26 (state 10)	[mutex = 0]
 15:	proc  1 (Producer:1) producers_consumers.pml:28 (state 11)	[items = (items+1)]
              Produce nuevo item
 17:	proc  1 (Producer:1) producers_consumers.pml:12 (state 1)	[printf('Produce nuevo item\\n')]
 19:	proc  3 (Consumer:1) producers_consumers.pml:38 (state 1)	[(items)]
 20:	proc  3 (Consumer:1) producers_consumers.pml:39 (state 2)	[((holes<255))]
 20:	proc  3 (Consumer:1) producers_consumers.pml:40 (state 3)	[items = (items-1)]
 22:	proc  3 (Consumer:1) producers_consumers.pml:44 (state 5)	[(!(mutex))]
 22:	proc  3 (Consumer:1) producers_consumers.pml:45 (state 6)	[mutex = 1]
                      Extrae elemento del buffer
 24:	proc  3 (Consumer:1) producers_consumers.pml:47 (state 8)	[printf('Extrae elemento del buffer\\n')]
 26:	proc  3 (Consumer:1) producers_consumers.pml:49 (state 9)	[mutex = 0]
 28:	proc  3 (Consumer:1) producers_consumers.pml:51 (state 10)	[holes = (holes+1)]
                      Consume elemento
 30:	proc  3 (Consumer:1) producers_consumers.pml:52 (state 11)	[printf('Consume elemento\\n')]
spin: trail ends after 30 steps
#processes: 4
		holes = 255
		items = 0
		mutex = 0
 30:	proc  3 (Consumer:1) producers_consumers.pml:34 (state 12)
 30:	proc  2 (Consumer:1) producers_consumers.pml:34 (state 12)
 30:	proc  1 (Producer:1) producers_consumers.pml:14 (state 5)
 30:	proc  0 (Producer:1) producers_consumers.pml:14 (state 5)
4 processes created
\end{lstlisting}

\textbf{Podemos comprobar que hay un ciclo, donde se reparten la ejecución el proceso 1 y 3 ; el proceso 0 y 2 permanecen en starvation}
Podemos solucionar esto con una cola o un array de procesos, en el cual se entra cuando se solicita el mutex, y este se va dando en orden, para así ningún proceso caiga en starvation.

\subsubsection{Un solo proceso, Productor}
\begin{lstlisting}[frame=single]  % Start your code-block

\end{lstlisting}

State-vector 28 byte, depth reached 7, errors: 0
        4 states, stored (8 visited)
        2 states, matched
       10 transitions (= visited+matched)
        2 atomic steps
hash conflicts:         0 (resolved)
\end{lstlisting}
\textbf{En este caso particular no hay starvation para Productor, Si que lo hay para el consumidor, porque no llega a ejecutarse nunca.}

\subsubsection{Tamaño de Buffer 0}
Hay starvation, ningún proceso entra nunca en zona crítica
\end{document}          
